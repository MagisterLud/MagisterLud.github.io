\documentclass{beamer}
\setbeamertemplate{footline}[frame number]
\mode<presentation>
{
\usetheme{Warsaw}
% or ...
%\usecolortheme{beaver}
%  \setbeamercovered{transparent}
% or whatever (possibly just delete it)
}

\usepackage{mathtext}
\usepackage[cp1251]{inputenc}
\usepackage[english,russian]{babel}
\usepackage[T2A]{fontenc}

\usepackage{inputenc}
\usepackage{amsmath}
\usepackage{amsthm}

\usepackage{amssymb}
\usepackage{graphicx}
%\usepackage{pscyr}

\usepackage{times}

% Or whatever. Note that the encoding and the font should match. If T1
% does not look nice, try deleting the line with the fontenc.

\usepackage{amscd}
\usepackage{euscript}
\usepackage{graphics}

\usepackage{inputenc}
\usepackage{mathtext}
%\usepackage{mathkerncmssi8}
%\pdfmapfile{+mathkerncmssi8.map}
\graphicspath{{./}{pics/}}
\usepackage[matrix,arrow,curve]{xy}
% ----------------------------------------------------------------
%\vfuzz2pt % Don't report over-full v-boxes if over-edge is small
%\hfuzz2pt % Don't report over-full h-boxes if over-edge is smafrll


% ----------------------------------------------------------------
% THEOREMS -------------------------------------------------------
\newtheorem{thm}{Theorem}
\newtheorem*{thm*}{Theorem}
\newtheorem{cor}{Corollary}
\newtheorem{lem}{Lemma}
%\newtheorem*{lem*}{Lemma}
%\newtheorem{prop}{Proposition}
\newtheorem{con}{Conjecture}
\newtheorem{prob}{Problem}
\newtheorem*{fr}{}
%
%\newtheorem{utv}{Óòâåðæäåíèå}
%\theoremstyle{definition}
\newtheorem{defn}{Definition}
%\newtheorem{ex}{Ïðèìåð}
%\newtheorem{cons}{Construction}
%\newtheorem*{cons*}{Construction}
%\newtheorem{rem}{Remark}
%\theoremstyle{remark}
%\newtheorem*{not*}{Notation}

\DeclareGraphicsExtensions{.pdf}
%\DeclareMathOperator{\conv}{conv}
%\DeclareMathOperator{\supp}{supp}
%\DeclareMathOperator{\des}{des}
%\DeclareMathOperator{\Sym}{Sym}
%\newcommand{\lk}{lk}
%\DeclareMathOperator{\dim}{dim}
%\DeclareMathOperator{\codim}{codim}
%\DeclareMathOperator{\cone}{Cone}

\usepackage{color}
\def\red#1{\textcolor{red}{#1}}
\def\blue#1{\textcolor{blue}{#1}}

%%%%\DeclareMathOperator{\lk}{lk}
%%%%\DeclareMathOperator{\st}{st}
\newcommand {\ib}[1]{\textit{\textbf{#1}}}
%\usetheme{Warsaw}
%\usetheme{Madrid} % My favorite!
%\usetheme{Boadilla} % Pretty neat, soft color.
%\usetheme{default}
\usetheme{Warsaw}
%\usetheme{Bergen} % This template has nagivation on the left
%\usetheme{Frankfurt} % Similar to the default
%with an extra region at the top.
%\usecolortheme{seahorse} % Simple and clean template
%\usetheme{Darmstadt} % not so good
% Uncomment the following line if you want %
% page numbers and using Warsaw theme%
%\setbeamertemplate{footline}[page number]
%\setbeamercovered{transparent}
%\setbeamercovered{invisible}
% To remove the navigation symbols from
% the bottom of slides%
%\setbeamertemplate{navigation symbols}{}
%\setlength{\parskip}{10pt plus 1pt minus 1pt}
\addtobeamertemplate{navigation symbols}{}{%
    \usebeamerfont{footline}%
    \usebeamercolor[fg]{footline}%
    \hspace{1em}%
    \insertframenumber/\inserttotalframenumber
}

\begin{document}

\title[$n$-valued groups and applications]{%
\huge {$n$-valued groups and applications}}
\author[V.\,M.\,Buchstaber]{\huge V.\,M.\,Buchstaber}
\vspace{10cm}
\institute{Steklov Mathematical Institute RAS\\
Steklov International Mathematical Center\\
HSE University, Int. Lab. of Algebraic Topology and Its Applications,\\
buchstab@mi-ras.ru}
\vspace{5cm}

\date{
\vspace{1cm}
\text{}\\
Euler International Mathematical Institute\\
St-Petersburg, 04 October, 2022}

%%%%%%%%%%%%%   1
\begin{frame}
  \maketitle
\end{frame}




%%%%%%%%%%%%%%   2
\begin{frame}{}
In various fields of research one encounters \\
a natural multiplication on a space, say $X$,  under which \\
the product of a pair of points is a subset of $X$\\ (for example, a finite subset).
\vspace{0.5cm}

The literature on multivalued groups and  their applications\\
is very large and includes titles from 19th century,\\
mainly in the context of hypergroups.
\end{frame}




%%%%%%%%%%%%%   3
\begin{frame}{}
In 1971 S.~Novikov and the author introduced a construction,\\
\text{suggested by the theory of characteristic classes of vector bundles},\\
in which the product of each pair of elements is an \blue{$n$-multiset},\\
i.e. the unordered set of $n$ points, possibly with repetitions.
\vskip.3cm
This construction leads to the notion of \blue{$n$-valued groups}.
\vskip.3cm
Soon afterwards the author gave an axiomatic definition\\
of $n$-valued groups and obtained the first results on their algebraic structure.\\
These results have found applications in the well-known problem of algebraic topology.
\end{frame}



%%%%%%%%%%%%   4
\begin{frame}{}
The condition of \blue{$n$-valuedness} is in fact very strong, so,
initially it seemed that the supply of interesting examples
of~$n$-valued groups is not very rich.
\vspace{0.5cm}

Soon afterwards the author developed the theory of formal,\\
or local, $n$-valued Lie groups,\\
which appeared to be rich of contents and have found important applications
in algebraic topology and theory of integrable systems.
\end{frame}




%%%%%%%%%%%%%%  5
\begin{frame}{}
In 1990 the author described the structure of an algebraic 2-valued group
on the complex projective line $\mathbb{C}P^1$.
\vspace{0.5cm}

Since 1993, Elmer Rees and the author collaborated\\
on the topological and algebraical theory of $n$-valued groups.
\vspace{0.5cm}

Since 1996 A.Veselov and the author started to work\\
on the applications of the $n$-valued groups theory\\
to the dynamical systems with discrete time.
\vspace{0.5cm}

At present, a number of authors successfully develope the theory
of $n$-valued groups (finite, discrete, topological, algebraic,
algebro-geometric) with applications in various areas\\
of mathematics and mathematical physics.
\end{frame}



%%%%%%%%%%%%%%%%   6
\begin{frame}{Symmetric powers of a space.}
For a topological space $X$, let $(X)^n$ denote its $n$-fold
symmetric power, i.e., $(X)^n=X^n/\Sigma_n$ where the symmetric group $\Sigma_n$ acts by permuting the coordinates.
\vspace{0.2cm}

An element of $(X)^n$ is called an $n$-subset of $X$ or just an $n$-set;\\
it is a subset with multiplicities of total cardinality $n$.
\vspace{0.2cm}

\underline{Example}. The spaces $(\mathbb{C})^n=\mathbb{C}^n/\Sigma_n$
and $\mathbb{C}^n$ are identified\\ using the map $\mathcal{S}:\mathbb{C}^n\to\mathbb{C}^n$
whose components are given by
\[
(z_1,z_2,\dots,z_n)\to e_r(z_1,z_2,\dots,z_n),\; 1\leqslant r\leqslant n,
\]
where $e_r$ is the $r$-th elementary symmetric polynomial.
\vspace{0.2cm}

The \blue{projectivisation} of the map $\mathcal{S}$ induces a
homeomorphism between $(\mathbb{CP}^1)^n$ and $\mathbb{CP}^n.$
\end{frame}




%%%%%%%%%%%%%%%%%%%%%%   7
\begin{frame}{$n$-valued group structure.}
An \blue{$n$-valued multiplication} on $X$ is a map
\[
\mu: X\times X \to (X)^n\,:\,\mu(x,y)=x*y=[z_1,z_2,\dots, z_n], \; z_k=(x*y)_k.
\]
\underline{Associativity}. The $n^2$-sets:
\begin{gather*}[x*(y*z)_1,x*(y*z)_2,\dots,x*(y*z)_n],\\
[(x*y)_1*z,(x*y)_2*z,\dots,(x*y)_n*z]
\end{gather*}
are equal for all $x,y,z\in X$.\\
\vspace{0.2cm}

\underline{Unit}. $e\in X$ such that\; $e*x=x*e=[x,x,\dots,x]$\; for all\; $x\in X$.\\
\vspace{0.2cm}

\underline{Inverse}. A map $\mathrm{inv}\colon X\to X$ such that
\vspace{-0.4cm}
\[
e\in \mathrm{inv}(x)* x\; \text{ and }\; e \in x*\mathrm{inv}(x)\;\text{ for all }\; x\in X.
\]
\vspace{-0.5cm}
\begin{fr}
The map $\mu$ defines an $n$-valued group structure on $X$ \\
if it is associative, has a unit and an inverse.
\end{fr}
\end{frame}




%%%%%%%%%%%%%%%   8
\begin{frame}{First results}
\begin{lem}
For each $m\in\mathbb{N}$, an $n$-valued group on $X$, with\\ the multiplication $\mu$, can be regarded as an
$mn$-valued group\\ by using as the multiplication the composition
\vspace{-0.2cm}

\[
X\times X\overset{\mu}\longrightarrow(X)^n\overset{(D)^m}{\longrightarrow} (X)^{m n},\;\text{ where }\; D\;\text{ is diagonal}.
\]
\end{lem}
\begin{defn}
A map $f\colon X\to Y$ is a \blue{homomorphism} of $n$-valued groups if
 \vspace{-0.2cm}

\[
f(e_X)=e_Y,\; f(\mathrm{inv}_X(x))=\mathrm{inv}_Y(f(x)))\;\text{ for all }\; x\in X,
\]
\[
\mu_{Y}(f(x),f(y))=(f)^n\mu_X(x,y)\;\text{ for all }\; x,y\in X.
\]
\end{defn}
\begin{lem}
Let $f\colon X\to Y$ be a homomorphism of $n$-valued groups. Then
\vspace{-0.2cm}

\[
\mathrm{Ker}(f)=\{x\in X\mid f(x)=e_Y\}\;\text{ is an $n$-valued group}.
\]
\end{lem}
\end{frame}




%%%%%%%%%%%%%%%   9
\begin{frame}{$2$-valued group structure on $\mathbb{Z}_{+}$.}
Consider the semigroup  of nonnegative integers $\mathbb{Z}_+$.\\[3pt]
Define the multiplication \;$\mu\colon \mathbb{Z}_+\times \mathbb{Z}_+\to (\mathbb{Z}_+)^2$ \\[3pt]
by the formula\; $x*y=[x+y,|x-y|]$.\\[7pt]

\underline{The unit}: $e=0$.
\vspace{0.2cm}

\underline{The inverse}: $\mathrm{inv}(x)=x$.
\vspace{0.2cm}

\underline{The associativity}:\\[3pt]
one has to verify that the $4$-subsets of
$\mathbb{Z}_+$
\[[x+y+z, |x-y-z|,x+|y-z|, |x-|y-z||]\]
and
\[[x+y+z,|x+y-z|,|x-y|+z,||x-y|-z|]\]
are equal for all nonnegative integers $x,y,z$.
\end{frame}




%%%%%%%%%%%%%%%  10
\begin{frame}{Additive $n$-valued group structure on $\mathbb{C}$.}
Define the multiplication\; $\mu\colon \mathbb{C}\times \mathbb{C}\to \mathbb{(C)}^n$\;
by the formula
\vspace{-0.2cm}

\[
x*y=[(\sqrt[n]{x}+\varepsilon^r \sqrt[n]{y}\,)^n,\quad 1\leqslant r\leqslant n],
\]
where $\varepsilon \in \mathbb{Z}_n$ is a
primitive $n$-th root of unity.
\vspace{0.2cm}

\underline{The unit}: $e=0$.
\vspace{0.2cm}

\underline{The inverse}: $\mathrm{inv}(x)=(-1)^n x$.
\vspace{0.2cm}

The multiplication is described by the polynomial equations
\vspace{-0.2cm}

\[
p_n(x,y,z)=\prod_{k=1}^n\big(z-(x*y)_k\big) = 0.
\]
For instance,
\vspace{-0.3cm}

\[
p_1=z-x-y,\quad p_2=(z+x+y)^2-4(x y+y z+z x),
\]
\[
p_3=(z-x-y)^3-27xyz.
\]
\end{frame}




%%%%%%%%%%%%%%   11
\begin{frame}{Coset and double coset groups.}
Let $G$ be a ($1$-valued) group with the multiplication $\mu_0$,\\
the unit $e_G$, and $\mathrm{inv}_G(u) = u^{-1}$.\\[5pt]
Let $A$ be a group  with $\# A=n$\; and\; $\varphi : A \to Aut\,G$ be\\
a homomorphism to the group of automorphisms of $G$. \\[5pt]
Denote by $X$ the quotient space $G/\varphi(A)$ of $G$ by the action\\ of the group
$\mathrm{Im}\varphi$, and denote by $\pi\colon G\to X$ the quotient map.\\[5pt]

Define the $n$-valued multiplication\; $\mu\colon X\times X\to (X)^n$\\ by the formula
\[
\mu(x,y)=[\pi(\mu_0(u,v^{a_i})),\quad 1\leqslant i\leqslant n,\quad a_i\in A],
\]
where\; $u\in \pi^{-1}(x),\;v\in \pi^{-1}(y)$\; and\; $v^a$\, is the image\\
of the action of $\varphi(a) \in Aut\,G, \; a \in A$\; on\; $G$.
\end{frame}





%%%%%%%%%%%%%%%%%   12
\begin{frame}{}
\begin{thm}
The multiplication $\mu $ defines an $n$-valued group structure\\
on $X=G/\varphi(A)$, called the \blue{coset group} of  $(G,A,\varphi)$,\\
with \underline{the unit} $e_X=\pi(e_G)$ and \underline{the inverse} $\mathrm{inv}_{X}(x)=\pi(\mathrm{inv}_{G}(u))$, where
$u\in\pi^{-1}(x)$.
\end{thm}
In case of $\ker \varphi = 0$ we will identify $A$ with $\varphi(A)
\subset Aut\, G$.
\vspace{0.2cm}

Let $H \subset G$ be a subgroup, and $\#H=n$.\\[3pt]
Denote by $X$ the space of double cosets $H \backslash G/H$.\\[3pt]
Define the $n$-valued multiplication\; $\mu\colon X\times X\to (X)^n$\\
by the formula
\[
\mu(x,y)=\{Hg_1H\}*\{Hg_2H\} = [\{Hg_1hg_2H\} : \; h \in H].
\]
\end{frame}





%%%%%%%%%%%%%%%   13
\begin{frame}{}
\begin{thm}
The multiplication $\mu $ defines an $n$-valued group structure\\
on $X=H \backslash G/H$, called a \blue{double coset group} of $(G,H)$,\\
with \underline{the unit} $e_X=\{H\}$ and \underline{the inverse} $\mathrm{inv}_{X}(x)=\{Hg^{-1}H\}$,\\ where $x=\{HgH\}$.
\end{thm}
Each coset group of $(G,A,\varphi)$, admits a double coset realization on $X=A \backslash G'/A$ as a double
coset group of $(G',A)$,\\ where $G'$ is the semidirect product of the groups $G$\, and\, $A$\\ with respect to the action of~$A$\, on\, $G$\; by means of $\varphi$.
\end{frame}





%%%%%%%%%%%%%%%%   14
\begin{frame}{Examples of the coset groups.}
\bf{(1)} The $2$-valued group $(\mathbb{Z},\mathbb{Z}_2,\varphi)$ on $\mathbb{Z}_+$.\\[5pt]

\bf{(2)} The additive $n$-valued group $(\mathbb{C},\mathbb{Z}_n,\varphi)$ on $\mathbb{C}$.\\[5pt]

\bf{(3)} Let $G$ be the infinite dihedral group
\[
G=\{a,b\mid a^2=b^2=e\}.
\]
\text{The interchange of $a$ and $b$ generates the automorphism group $A$,}
$\# A=2$. Then
\vspace{-0.3cm}

\[
\qquad X=G/A=\{u_{2n}, u_{2n+1}\}, \quad n\geqslant 0,
\]
where\; $u_{2n} = \{(ab)^n, (ba)^n\}$,\quad $u_{2n+1} = \{b(ab)^n, a(ba)^n\}$.\\[3pt]
Then the multiplication is given by the formula
\vspace{-0.2cm}

\[
u_k*u_{\ell}=[u_{k+\ell},u_{|k-\ell|}].
\]
Thus $X$ is isomorphic to the $2$-valued group on~$\mathbb{Z}_+$.
\end{frame}




%%%%%%%%%%%%%%%   15
\begin{frame}{Examples of the coset groups.}
\bf{(4)} Let $G$ be a finite group, $\# G=n$.\\ Let $A=G$ acts by inner automorphisms
\[
g^a=a^{-1}g a,\quad g\in G,\; a\in A.
\]
Thus the \blue{set of characters} of $G$ is an $n$-valued coset group\\ on $X=G/A$.
\vspace{0.2cm}

Consider $G=\Sigma_3$.\\[6pt]
Then $X=\{e, x_1,x_2\}$ is a $6$-valued group:
\begin{align*}
x_1*x_1 &= [e,e,e,x_1,x_1,x_1],\\
x_1*x_2=x_2*x_1 &= [x_2,x_2,x_2,x_2,x_2,x_2],\\
x_2*x_2 &= [e,e,x_1,x_1,x_1,x_1].
\end{align*}

Note, that this $6$-valued group on three elements is impossible to reduce to a $n$-valued group with $n<6$.
\end{frame}





%%%%%%%%%%%%%%%%%   16
\begin{frame}{Examples of the coset groups.}
\bf{(5)}  Consider the $n$-fold direct product $G^n$ of a group~$G$ by itself.\\
The group $\Sigma_n$ acts on~$G^n$ by permuting the factors.
\begin{fr}
Therefore, for any group~$G$, the symmetric product $(G)^n$ is endowed with the structure of an $n!$-valued coset group.
\end{fr}
If~$G$ is commutative, with the operation  $\mu(g',g'') = g'+g''$,\\
then we have an $n!$-valued group homomorphism
\[ (\mu)^n : (G)^n \longrightarrow G, \quad [g_1, \ldots, g_n] \longrightarrow g_1+ \cdots + g_n, \]
where $G$ is treated as an $n!$-valued group with the diagonal
operation $\mu(g',g'') = [g'+g'', \ldots, g'+g'']$.\\[5pt]
In this way we obtain the $n!$-valued group $\mathrm{Ker}(\mu)^n$.
\end{frame}




%%%%%%%%%%%%%%   17
\begin{frame}{}
Take a smooth elliptic curve. It equips the torus  $T^2$\\ with a commutative group structure. \\
The construction above produces a structure\\ of an $n!$-valued group on $(T^2)^n$ for each~$n$.
\begin{fr}
Thus this construction produces a structure\\ of an $n!$-valued group on the complex projective space\\[3pt]
\qquad \qquad $\mathbb{C}P^{n-1}=\mathrm{Ker}((\mu)^n : (T^2)^n \to T^2)$.
\end{fr}

For $n=2$, this yields a structure of a $2$-valued group on $\mathbb{C}P^1$.
\begin{fr}
Using the automorphisms of a smooth elliptic curve, we obtain\\
$2$-valued and $3$-valued coset group structures on $\mathbb{C}P^1$.
\end{fr}
\end{frame}




%%%%%%%%%%%%%%%%%%   18
\begin{frame}{A family of non-coset groups.}
Consider the $(2k+1)$-valued group on $X(3)=\{x_0=e,x_1,x_2\}$ where the  multiplication is given by the formulas
\begin{align*}
x_1*x_1 &= [\underbrace{x_1,\dots,x_1}_{k},\underbrace{x_2,\dots,x_2}_{k+1}],\\[7pt]
x_1*x_2=x_2*x_1 &= [e,\underbrace{x_1,\dots,x_1}_{k},\underbrace{x_2,\dots,x_2}_{k}],\\[7pt]
x_2*x_2 &= [\underbrace{x_1,\dots,x_1}_{k+1},\underbrace{x_2,\dots,x_2}_{k}].
\end{align*}
\begin{thm}[S. Evdokimov, 2005]
The $(2k+1)$-valued group $X(3)$ is a coset group if and only if $2k+1=p^s$, where $p$ is a prime number.
\end{thm}

\end{frame}





%%%%%%%%%%%%%%   19
\begin{frame}{$n$-valued groups on sets\\ of irreducible unitary representations of groups.}
Let~$G$ be a finite group, $\# G=m$\; and\; let $\rho_0, \rho_1, \ldots, \rho_k$ be\\
the set of all its irreducible unitary representations,\\
where $\rho_0$ is the trivial one-dimensional representation.
\vspace{0.2cm}

Consider the decomposition of tensor products of irreducible representations in direct sums of irreducible representations
\[
\rho_i \otimes \rho_j = \rho_i \rho_j = \sum_{l=0}^k a_{ij}^l \rho_l
\]
where $a_{ij}^l$ is the multiplicity of the representation $\rho_l$\\ in the product $\rho_i \rho_j$.
\vspace{0.2cm}

We have $a_{ij}^l=a_{ji}^l$ and, as the classical theory implies,
\[
a_{ij}^0 =
\begin{cases}
  1, & \mbox{if } \rho_j=\bar{\rho_i} \\
  0, & \mbox{otherwise}.
\end{cases}
\]
\end{frame}




%%%%%%%%%%%%%   20
\begin{frame}{}
Let us denote the dimension of $\rho_l$ by $d_l$. We have $\sum\limits_{l=0}^k d_l^2=m$.\\[3pt]

\text{Let $n$ be the least common multiple (LCM) of $d_id_j, \; 0 \leqslant i \leqslant j \leqslant k$.} \\[3pt]
Introduce the set of integers $m_{ij}^l = n a_{ij}^l\, \dfrac{d_l}{d_id_j}$.\; We have
\vspace{-0.4cm}

\[
 \sum_{l=0}^k m_{ij}^l = n.   \qquad \qquad \qquad {}
 \]
\vspace{-0.2cm}

Set $x_l = \frac{1}{d_l}\rho_l$\; and consider the set $X=\{x_0,\ldots,x_k \}$.
\begin{thm}
The tensor product of representations defines on~$X$ a structure
of an $n$-valued group with the product  $\mu : X \times X \to (X)^n$, where\\
$\mu(x_i,x_j)=x_i*x_j$ is the $n$-multiset containing the element $x_l$\\
with the multiplicity $m_{ij}^l $,\, the element $x_0$ is the unit,\\
and the inverse $\mathrm{inv} : X \to X$ is given by the complex conjugation map,\;
i.e., $\mathrm{inv}(x_l)=\bar{x}_l$, where $\bar{x}_l=\frac{1}{d_l}\bar{\rho}_l$.
\end{thm}
\end{frame}




%%%%%%%%%%%%   21
\begin{frame}{\textbf{Example.} $G=\Sigma_3$.}
There are irreducible representations $\rho_0,\; \rho_1,\; \rho_2$,\\
of dimensions $d_0=1, \; d_1=1$, and $d_2=2$, respectively,\\ with the tensor product table
\vspace{-0.3cm}
\begin{gather*}
\rho_0 \rho_l = \rho_l, \quad l=0,1,2, \\
\rho_1^2=\rho_0, \quad \rho_1 \rho_2=\rho_2, \quad \rho_2^2=\rho_0 \oplus \rho_1 \oplus \rho_2.
\end{gather*}
So $n=d_2^2=4$, and on the set $X= \{ x_0=\rho_0, x_1=\rho_1, x_2=\frac{1}{2}\rho_2 \}$
we obtain  a 4-valued group with the multiplication
\vspace{-0.3cm}
\begin{align*}
\qquad x_1*x_1 &=[x_0,x_0,x_0,x_0], \\
\qquad x_1*x_2 &=[x_2,x_2,x_2,x_2], \\
\qquad x_2*x_2 &=[x_0,x_1,x_2,x_2],
\end{align*}
\vskip-0.2cm
the unit $e=x_0$, and the identity map $\mathrm{inv}$.
\begin{fr}
Note that in this case, the 4-valued group structure\\
cannot be replaced by a less valued structure.
\end{fr}
\end{frame}





%%%%%%%%%%%%%%%%%   22
\begin{frame}{Generalization of Pontryagin's duality\\ to the case of non-commutative groups.}
In the case of a commutative group~$G$, the identification\\ of
irreducible unitary representations with the conjugacy classes
of~$G$ corresponds to the Pontryagin duality.
\vspace{0.4cm}

For a general finite group~$G$, with $\#G=m$, we obtain on these isomorphic sets two
structures: the structure of $m$-valued groups and the structure of $n$-valued groups where $n=\text{LCM}\{ d_id_j \}$.\\[2pt]
\begin{fr}
For a given noncommutative finite group, this construction yields an n-valued group dual to it.
\end{fr}
For example, for $G=\Sigma_3$ we have $m=6$ and $n=4$.\\ Thus, we obtain a 4-valued group dual to the 1-group $\Sigma_3$.
\end{frame}




%%%%%%%%%%%%%%   23
\begin{frame}{Action on a space.}
An $n$-valued group $X$ \blue{acts on a space} $Y$ if there is a mapping
\[
\phi\colon X\times Y\to (Y)^n,
\]
also denoted $x\circ y=\phi(x,y)$, such that the two $n^2$-subsets of $Y$
\[
x_1\circ(x_2\circ y)\;\text{ and }\;(x_1*x_2)\circ y
\]
are equal for all $x_1,x_2\in X$ and $y\in Y$;\\[4pt]
and also
\[
e\circ y=[y,y,\dots,y]\; \text{ for all } y\in Y.
\]
\end{frame}




%%%%%%%%%%%%%%%%%%  24
\begin{frame}{The coset and double coset actions.}
Let $G$ be a certain 1-valued group and $\varphi : A \to Aut\,G$, $\# A=n$.\\
Suppose that $G$ and $A$ \blue{equivariantly} act  on some space $V$, i.e.
\[
(g(v))^a= g^a(v^a),\; \text{where} \; a\in A, \; g\in G, \; v\in V.
\]
\text{There is a natural action of the~$n$-valued coset group $X=G/\varphi(A)$}
on the space of orbits $Y=V/A$.
\vspace{0.2cm}

Let~$G$ be a group and let $H \subset G$ be a subgroup, $\# H=n$.\\
Suppose~$G$ acts on a space~$V$.\; Then the double coset group $X=H \backslash G/H$ acts on~$Y=V/H$ according to the formula
\[
\{HgH\}\{Hv\} = [H(gh)v : h \in H].
\]
\end{frame}



%%%%%%%%%%%%%%%%%   25
\begin{frame}{Algebraic action.}
For a given action
\[
\phi\colon X\times Y\to (Y)^n,
\]
\text{define $\Gamma_x$, the \blue{graph of the action} of $x \in X$, as the subset of $Y \times Y$,}
which consists of the pairs $(y_1,y_2)$ such that $y_2\in \phi(x, y_1)$.
\begin{block}{Definition}
The action of an~$n$-valued group~$X$ on an~algebraic variety $M$\\
is called \blue{algebraic} if the action of any element of $X$\\
is determined by an algebraic correspondence,\\
i.e., its graph is an \blue{algebraic subset} in $M \times M$.
\end{block}
\end{frame}




%%%%%%%%%%%%%%%%   26
\begin{frame}{}
\text{It is well known that treating an action of a group~$G$ on a space~$V$}
as a dynamical system leads to fruitful investigations\\ in ergodic theory and geometry.
\vskip.2cm
Among the directions of these investigations, the study\\ of dynamical systems with discrete time,
whose evolution\\ is described by a map of~$V$ into itself, plays a distinguished role.
\vskip.2cm
It is natural to apply the theory of $n$-valued groups to $n$-valued analogues of these systems.
\begin{block}{Definition}
An \blue{$n$-valued dynamics} $T$ with \blue{discrete time} on a space~$Y$\\ is a continuous map $T : Y \to (Y)^n$.
\end{block}
Thus, if we consider~$Y$ as a space of states, then an $n$-valued
dynamics $T : Y \to (Y)^n$ determines possible states $T(y) = [y_1,
\ldots, y_n]$ at the moment $(t+1)$ as functions\\ of the state $y$ of
the system at the moment~$t$.
\end{frame}




%%%%%%%%%%%%%%%%%   27
\begin{frame}{Example.}
Let\; $T(x,y)=b_0(x)y^n+b_1(x)y^{n-1}+\cdots+b_n(x).$\\[5pt]
The equation $T(x,y)=0$ defines an~$n$-valued map\\ (or a $n$-valued dynamics)
$\mathbb{C}\to\mathbb{C}$ under which $x$ is taken\\ to the set of roots $[y_1,y_2,\dots, y_n]$ of equation $T(x,y)=0$.
\vskip.4cm
In general case the number of different images of a point grows exponentially with the number of iterations of the map.
\vskip.4cm
In exceptional case the growth is polynomial.

\end{frame}




%%%%%%%%%%%%%%%%%   28
\begin{frame}{}
The pictures demonstrates the difference between
exceptional and general situations.
\vskip .5cm

\parbox[r][5cm][b]{8cm}{\hspace*{0.4cm}\includegraphics[totalheight=5cm]{fig1.jpg}}
\end{frame}




%%%%%%%%%%%%%%%%%   29
\begin{frame}{The Euler-Chasles correspondence.}
The polynomial
\[
T(x,y)= A x^2 y^2  + Bxy(x + y) + C(x^2  + y^2 ) + Dxy + E(x + y) + F,
\]
where $A,B,C,D,F$ are constants, defines the $2$-valued dynamics,\\
in which the number of different images after the $k$-th iteration is $k+1$,
\blue{but not} $2^k$ as one could expect.
\end{frame}





%%%%%%%%%%%%%%%   30
\begin{frame}{}
The picture explains this fact as the curve $T(x,y)=0$\\
describes the geometric situation in the famous \blue{Poncelet porism}\\
for two conics on the plane.
\vskip 1.5cm

\parbox[r][5cm][b]{3.5cm}{\hspace*{1.5cm}
\includegraphics[height=6cm]{fig2.jpg}}
\end{frame}





%%%%%%%%%%%%%%%%%   31
\begin{frame}{}
It is known that for Euler-Chasles correspondence there exists\\
an even elliptic function $f(z)$ of the degree~$2$,\\
such that if $x=f(z)$ then $[y_1,y_2]=[f(z+a),f(z-a)]$ for some $a$.
\vskip.3cm
This means that the Euler-Chasles correspondence is\\
the projection of the mapping $z \to z+a$ of the \blue{elliptic curve} $E$\\
\text{into itself to the projective line~$\mathbb{CP}^1$ which is a coset space $E/\mathbb{Z}_2$,}
where $\mathbb{Z}_2$ is acting on $E$ as $z \to -z$.
\vskip.3cm
Thus, we have the representation of the two-valued group
$\mathbb{Z}_{+}=\mathbb{Z}/\mathbb{Z}_2$ with the multiplication
\vspace{-0.3cm}

\[
x*y=[x+y,|x-y|].
\]
\begin{thm} [V.B., A.Veselov, 1996]
All algebraic actions of the two-valued group $\mathbb{Z}_{+}$ on $\mathbb{CP}^1$\\
are generated either by the Euler-Chasles correspondence\\
or by a reducible correspondence.
\end{thm}
\end{frame}





%%%%%%%%%%%%%%%%%%   32
\begin{frame}{Integrable $n$-valued dynamics.}
\begin{block}{Definition}
An $n$-valued dynamics~$T$ with discrete time on a space~$Y$\\
is said to be \blue{integrable} by means of an $n$-valued group~$X$\\
with a single generator $a$ if the embedding $i : Y \to X \times Y$, \\
$i(y)=(a,y),$ extends to an action $\varphi$ of the $n$-valued group~$X$ on~$Y$, i.e., the triangle
\vspace{-0.5cm}

$$
\xymatrix{
Y \ar[r]^T \ar[d]^i & (Y)^n \\
X \times Y \ar[ur]_{\varphi} & }
$$
\vspace{-0.5cm}

\quad is commutative.
\vspace{0.2cm}
\end{block}
\end{frame}





%%%%%%%%%%%%%%%%   33
\begin{frame}{Singly generated $n$-valued groups.}
For 1-valued groups, the only singly generated group are
 $\mathbb{Z}$ and its quotients, the cyclic groups
$\mathbb{Z}_m$.\\
The variety of singly generated $n$-valued groups is much wider.
\begin{block}{Definition}
An $n$-valued group $(X,*,e,\mathrm{inv})$  is said to be \blue{ singly generated}
if there is an element $g \in X$ such that each element $x\in X$ belongs
to the multiset $g^k*\bar{g}^l$ for some $k$ and $l$, where $\bar{g}=\mathrm{inv}(g)$.
\end{block}
Let $(X,*,e,\mathrm{inv})$ be a singly generated $n$-valued group with\\
a generator $g$ such that $\bar{g}\in g^k$ for some $k\geqslant 1$.
\vskip.2cm
Define a graph $\Gamma_g$ in the following way:\\
 - any $x\in X$ gives a node in $\Gamma_g$;\\
 - there is an arrow of weight $l$ from $x_1$ to $x_2$ if $x_2$ belongs\\
 to the multiset $g*x_1$ with multiplicity $l$.
\vskip.2cm
For example, there is an arrow of multiplicity $n$ from $e$ to $g$.
\end{frame}




%%%%%%%%%%%%%%%%%%%   34
\begin{frame}{Finite groups with an irreducible exact representation.}
William Burnside (1852--1927)
\begin{thm}[W. Burnside]
Let $\rho$ be an irreducible faithful representation of a
finite group~$G$.

Then each irreducible representation of~$G$ enters\\
the decomposition of a power $\rho^k=\rho \otimes \cdots \otimes\rho$ into the sum\\
of irreducible summands, for some $k$.
\end{thm}
\begin{cor}
If a finite group~$G$ possesses a \blue{faithful irreducible} representation,
then the $n$-valued group on the set of its irreducible representations is
\blue{singly generated}.
\end{cor}
\end{frame}





%%%%%%%%%%%%%%%%%%   35
\begin{frame}{Construction of integrable $n$-valued
dynamical systems with discrete time.}
\begin{fr}
Any \blue{simple} finite group possesses a faithful irreducible
representation.
\end{fr}
For example, for $G=\Sigma_3$ the $2$-dimensional
irreducible representation $\rho_2$ is faithful, and the 4-valued
group\\ on $X=(e,x_1,x_2)$ is singly generated, namely,
$x_2*x_2=(e,x_1,x_2,x_2)$.
\vskip.3cm
A general condition for the existence of faithful irreducible representations
for finite groups see in the paper W.Gaschutz (1954).
\begin{fr}
Singly generated $n$-valued groups define integrable $n$-valued
dynamical systems with discrete time.
\end{fr}
\end{frame}




%%%%%%%%%%%%%%%%%%%   36
\begin{frame}{$n$-Hopf algebras.}
Associate to a space $X$ the ring $\mathbb{C}[X]$ of all continuous complex valued functions on~$X$.
For any positive integer $k$, we have\\ the canonical map
\vspace{-0.4cm}

\[
s_k : \mathbb{C}[X] \longrightarrow \mathbb{C}[(X)^k],
\]
such that  $s_k(f)[x_1, \ldots,x_k] = \sum_{i=1}^k f(x_i)$.
\begin{block}{Definition}
Let~$X$ be an~$n$-valued group with the multiplication
$\mu : X \times X \longrightarrow (X)^n$.\; The \blue{diagonal map}
\[
\Delta : \mathbb{C}[X] \longrightarrow \mathbb{C}[X \times X]
\approx \mathbb{C}[X] \widehat{\otimes} \mathbb{C}[X]
\]
is the linear map $\Delta = \frac{1}{n}F$, where
\vspace{-0.5cm}

\[
F(f)(x,y) = s_n(f)(\mu(x,y)) = \sum_{i=1}^n f(z_i)
\]
\vspace{-0.5cm}

and $\mu(x,y) = x*y = [z_1, \ldots, z_n]$.
\end{block}
\end{frame}




%%%%%%%%%%%%%%%%%   37
\begin{frame}{}
For $n=1$ the diagonal map $\Delta$ allows one to introduce\\ a Hopf algebra structure on $\mathbb{C}[X]$.
\begin{lem}
The ring of functions $\mathbb{C}[X]$ on an $n$-valued group $X$\\ is a coalgebra $(\mathbb{C}[X],\Delta,\varepsilon)$, where
$\Delta$ is the introduced above diagonal map and the counit $\varepsilon : \mathbb{C}[X] \to \mathbb{C}$ is induced\\ by the unit $e \to X$.
\end{lem}
An action $\varphi : X \times Y \to (Y)^n$ of an $n$-valued group~$X$ on~$Y$\\
defines a comodule $(\mathbb{C}[Y],\Delta_Y)$ over the coalgebra $(\mathbb{C}[X],\Delta,\varepsilon)$ where
\[
\Delta_Y : \mathbb{C}[Y] \longrightarrow \mathbb{C}[X \times Y] \approx \mathbb{C}[X] \widehat{\otimes} \mathbb{C}[Y],
\]
\[
\Delta_Y(g)(x,y) = \frac{1}{n}\sum_{i=1}^n g(y_i)\; \text{ and }\; [y_1, \ldots, y_n]= \varphi(x,y).
\]
\end{frame}





%%%%%%%%%%%%%%%%%   38
\begin{frame}{}
The $\mathrm{inv} : X \to X$ map axiom implies that there is a map
$\mathrm{inv}^\bot : X \to (X)^{n-1}$ such that the diagram
$$
\xymatrix{
X \ar[r]^-d \ar[rrd]_{\mathrm{inv}^\bot} & {X \times X \;\;} \ar[r]^{1\times \mathrm{inv}} & {\;\; X \times X} \ar[r]^{\;\mu} & (X)^n \\
&& {(X)^{n-1} \quad} \ar[ru]_{i_n} &}
$$
is commutative; here $i_n[x_1,\ldots,x_{n-1}] = [x_1, \ldots,x_{n-1},e]$.
\vskip.3cm
This diagram states that the homomorphism
$s_n(\cdot)(\mu(x,\mathrm{inv}\,x)) : \mathbb{C}[X] \to
\mathbb{C}[X]$ is split into composition\\ of homomorphisms
$(\mathrm{inv}^\bot)^* \,i_n^* \,s_n(\cdot)$.
\vskip.3cm
If $n=1$, then this algebraic condition determines the antipode
$$(\mathrm{inv})^* : \mathbb{C}[X] \to \mathbb{C}[X]$$ in the Hopf algebra $\mathbb{C}[X]$.
\end{frame}




%%%%%%%%%%%%%%%   39
\begin{frame}{}
The multiplication and the comultiplication in a Hopf algebra
are related by the fact that the diagonal map is an algebra
homomorphism.
\vskip.2cm
In papers by E.~Rees and the author, a new algebraic notion,
that of  an ring $n$-homomorphism, was introduced and\\
a definition of $n$-Hopf algebra was given in order to characterize this relation in the case of $n$-valued groups.
\vskip.2cm
Applications of $n$-Hopf algebras were developed, based on the following generalization of a classical result about Hopf algebras:
\begin{fr}
If $X$ is a topological $n$-valued group, then the ring $\mathbb{C}[X]$ and\\
the cohomology algebra $H^{2*}(X;\mathbb{C})$, in the case $H^{1+2*}(X;\mathbb{C})=0$,\\
are $n$-Hopf algebras.
\end{fr}
\begin{cor}
The spaces $\mathbb{C}P^m$ for $m>1$ do not admit the structure\\ of a 2-valued group.
\end{cor}
\end{frame}




%%%%%%%%%%%%%%%%   40
\begin{frame}{Frobenius recursion.}
Let $f: \; A\to B$ be a linear map, where $A$ and $B$ are commutative algebras with unit
over a field of characteristic 0.
\vskip.3cm
Define, by induction,  linear maps
\[ \Phi_n(f):\; A^{\otimes n} \longrightarrow B\]
starting with $\Phi_1(f) = f$
$$ \Phi_2(f)(a_1,a_2)=f(a_1)f(a_2)-f(a_1a_2)$$
and for $n\geqslant 2$
\begin{multline*}
\Phi_{n+1}(f)(a_1,\dots,a_{n+1})=f(a_1)\Phi_n(f)(a_2,\dots,a_{n+1})-\\[5pt]
-\sum_{l=2}^{n+1}\Phi_{n}(f)(a_2,\dots,a_1a_l,\dots,a_{n+1}).
\end{multline*}

\end{frame}




%%%%%%%%%%%%%%%%%   41
\begin{frame}{Frobenius $n$-homomorphisms.}
\begin{lem}
If $B$ is a domain and $\Phi_{n+1}(f)\equiv 0$ but
$\Phi_n(f)\not \equiv 0$ then $f(1)=n$.
\end{lem}
\begin{cor}
If $f\,:\;A\rightarrow B$ satisfies $\Phi_{n+1}(f)\equiv 0$ and
$B$ is a domain then $f(1)\in\{0,1,2,\ldots,n\}$
\end{cor}
\begin{block}{Definition (V.Buchstaber, E.Rees, 1997)}
A linear map  $f\,:\;A\rightarrow B$ is a \blue{Frobenius $n$-homomorphism} if
\[
\Phi_{n+1}(f)\equiv 0 \quad  \rm{and} \quad f(1)=n.
\]
\end{block}
\end{frame}




%%%%%%%%%%%%%%%   42
\begin{frame}{Examples.}
\underline{1-homomorphism }
\[
f(1)=1
\]
and
\[
f(a_1a_2)= f(a_1)f(a_2)
\]
that is,  a ring homomorphism.
\vskip.3cm
\underline{2-homomorphism }
\[
f(1)=2
\]
and
\begin{multline*}
2f(a_1a_2a_3) = f(a_1)f(a_2a_3)+f(a_1a_2)f(a_3)+\\
+f(a_2)f(a_1a_3)-f(a_1)f(a_2)f(a_3).
\end{multline*}
\end{frame}




%%%%%%%%%%%%%%%   43
\begin{frame}{Solution of the recursion.}
Let ${\mathcal D}_f(a_1,\ldots,a_{n+1})$ be the determinant of the matrix
\vspace{-0.4cm}

\bigskip
\[
\left(\begin{array}{llll}
 f(a_1)&1&\ldots&0 \\
  f(a_1a_2)&f(a_2)&\ldots&0\\
  \vdots&\vdots&&\vdots\\
  f(a_1\cdots a_n)&f(a_2\cdots a_n)&\ldots&n\\
   f(a_1\cdots a_{n+1})&f(a_2\ldots a_{n+1})&\ldots&f(a_{n+1})
  \end{array}\right)
  \]
\begin{thm}[V.Buchstaber, E.Rees, 1997]
\vspace{-0.3cm}

\[
\Phi_{n+1}(f)(a_1,\ldots,a_{n+1})=\frac{1}{(n+1)!}\sum\limits_{\sigma\in\Sigma_{n+1}}
{\mathcal D}_f\big(a_{\sigma(1)},\ldots,a_{\sigma(n+1)}\big).
\]
\end{thm}
\begin{cor}
The map $\Phi_{n+1}(f)$ is a symmetric multi-linear form.
\end{cor}
\end{frame}




%%%%%%%%%%%%%%%   44
\begin{frame}{Multiplicative property.}
We denote the sub-algebra of symmetric tensors in $A^{\otimes n}$ by
${\mathcal S}^n A$.
\vskip.3cm
A typical element of ${\mathcal S}^n A$ is
\[
\mathbf{a}=\sum_{\sigma\in\Sigma_n} a_{\sigma(1)}\otimes\cdots\otimes a_{\sigma(n)}
\]
and
\[
\mathbf{ab}=\sum_{\sigma_1,\sigma_2\in\Sigma_n} a_{\sigma_1(1)}b_{\sigma_2(1)}\otimes\cdots\otimes a_{\sigma_1(n)}b_{\sigma_2(n)}.
\]
\begin{thm}[V.Buchstaber, E.Rees]
The linear map $f:\;A\rightarrow B$ is a Frobenius $n$-homomorphism if and only if
\[ \frac{1}{n!} \Phi_n(f):\; {\mathcal S}^nA\rightarrow B \]
is a ring homomorphism.
\end{thm}
\end{frame}





%%%%%%%%%%%%%%%%%%%   45
\begin{frame}{References}
\begin{thebibliography}{99}%\tiny
\small{
\bibitem[Bu79]{Bu79} V.M.Buchstaber,
\newblock\emph{Characteristic cobordism classes and topological applications of the theories of one-valued and
two-valued formal groups.}
\newblock  J. Sov. Math. 11, 1979, 815-921.

\bibitem[Bu90]{Bu90} V.M.Buchstaber,
\newblock \emph{Functional equations that are associated with addition theorems for  elliptic functions, and two-valued algebraic groups.}
\newblock Russian Math. Surveys 45, N~3, 1990,  213--215.

\bibitem[Bu-Ve]{Bu-Ve96} V.M.Buchstaber, A.P.Veselov,
\newblock \emph{Integrable correspondences and algebraic representations of multivalued  groups.}
\newblock Internat. Math. Res. Notices, N~8, 1996, 381--400.

\bibitem[BVEP]{BVEP96} V.M.Buchstaber, A.M.Vershik, S.A.Evdokimov, I.N.Ponomarenko,
\newblock \emph{Combinatorial algebras and multivalued involutive groups.}
\newblock Funct. Anal. Appl. 30, N~3, 1996, 158--162.
}
\end{thebibliography}

\end{frame}





%%%%%%%%%%%%%%%%%%%   46
\begin{frame}{References}
\begin{thebibliography}{99}%\tiny
\small{
\bibitem[BuRe]{BuRe97} V.M.Buchstaber, E.G.Rees,
\newblock\emph{Multivalued groups, their representations and Hopf algebras.}
\newblock Transform. Groups, v.~2, N~4, 1997, 325--349.

\bibitem[BuDr]{BuDr18} V.M.Buchstaber, V.I.Dragovich,
\newblock \emph{Two-Valued Groups, Kummer Varieties, and Integrable Billiards.}
\newblock Arnold Math. J., 4:1 (2018), 27--57.

\bibitem[Bu-Ve]{Bu-Ve19} V.M.Buchstaber, A.P.Veselov,
\newblock \emph{Conway topograph, $PGL_2(\mathbb Z)$-dynamics and two-valued groups.}
\newblock Russian Math. Surveys, 74:3(447), 2019, 387--430.

\bibitem[BVG]{BVG22} V.M.Buchstaber, A.P.Veselov, A.A.Gaifullin,
\newblock \emph{Classification of involutive commutative two-valued groups.}
\newblock  Russian Math. Surveys, 77:4(466), 2022.
}
\end{thebibliography}

\end{frame}




%%%%%%%%%%%%%%%%%%%%   47
\begin{frame}
\begin{center}
\Huge \alert{Thank You for the Attention!}
\end{center}
\end{frame}


\end{document}


